% Basic packages
\usepackage[a4paper, margin=2cm]{geometry}
\usepackage{tikz}
\usetikzlibrary{calc, positioning, external, fit}
\tikzexternalize[prefix=figs/,optimize command away=\includepdf]
\tikzset{
	vector/.style = {#1, ultra thick, -stealth, cap=round},
}
\usepackage{hyperref}

% Colors
\definecolor{xred}{HTML}{BD4242}
\definecolor{xblue}{HTML}{4268BD}
\definecolor{xgreen}{HTML}{52B256}
\definecolor{xpurple}{HTML}{7F52B2}
\definecolor{xorange}{HTML}{FD9337}

% General defs
\newcommand\newterm[1]{\textbf{#1}}

% Math related defs
\usepackage{amsthm, commath, bm, mathtools, amsfonts}
\renewcommand\vec{\mathbf}
\newcommand\uvec[1]{\hat{\vec{#1}}}
\newcommand\bigO[1]{\mathcal{O}\left(#1\right)}
\newcommand\pnt[1]{\mathbf{#1}}
\newcommand{\eb}[1]{\mathbf{\hat{e}_{#1}}}

% Set math in section to be bold
\usepackage{etoolbox}
\makeatletter
% \tracingpatches
\patchcmd{\@sect}{#8}{\boldmath #8}{}{}
\let\ori@chapter\@chapter
\def\@chapter[#1]#2{\ori@chapter[\boldmath#1]{\boldmath#2}}
\makeatother

% Cancel terms
\usepackage[thicklines]{cancel}
\renewcommand\CancelColor{\color{xred}}
\newcommand{\cancelcol}[2][xred]{ % This is such a silly solution...
	\renewcommand\CancelColor{\color{#1}}
	\cancel{#2}
	\renewcommand\CancelColor{\color{xred}}
}

% Row- and column-vectors: arguments separated by ";".
% Example: $\vec{a} = \colvec{1;2;3;4}$.
\makeatletter
\newcommand\rcvector[2][\\]{\ensuremath{%
		\global\def\rc@delim{#1}%
		\negthinspace\begin{bmatrix}
			\rc@vector #2;\relax\noexpand\@eolst%
		\end{bmatrix}}}
\def\rc@vector #1;#2\@eolst{%
	\ifx\relax#2\relax
		#1
	\else
		#1\rc@delim
		\rc@vector #2\@eolst%
	\fi}
\makeatother
\newcommand{\colvec}{\rcvector}
\newcommand{\rowvec}[1]{\rcvector[,\;]{#1}}
\newcommand{\GenericRowVec}[2][n]{\rowvec{#2_{1};#2_{2};\dots;#2_{#1}}}
\newcommand{\GenericColVec}[2][n]{\colvec{#2^{1};#2^{2};\vdots;#2^{#1}}}

% Easier space-notation
\newcommand{\Rs}[1][]{\mathbb{R}^{#1}}
\newcommand{\Cs}[1][]{\mathbb{C}^{#1}}

% Better imaginary unit and natural base notation (to separate from variables)
\newcommand{\iu}{\mathrm{i}\mkern1mu}
\newcommand{\eu}{\mathrm{e}}
\newcommand{\Eu}[1]{\mathrm{e}^{#1}}
\newcommand{\EX}[1]{\exp\left(#1\right)}

% Norms and the like
\renewcommand{\norm}[1]{\left| #1 \right|}
\newcommand{\vnorm}[1]{\left| \vec{#1} \right|}
\newcommand{\snorm}[1]{\left| #1 \right|}
